\documentclass[runningheads]{llncs}
%
\usepackage[T1]{fontenc}
% T1 fonts will be used to generate the final print and online PDFs,
% so please use T1 fonts in your manuscript whenever possible.
% Other font encondings may result in incorrect characters.
\usepackage{multirow}
\usepackage{multicol}
\usepackage{makecell}
\usepackage{xcolor}
\usepackage{graphicx}
\usepackage{orcidlink}
\usepackage{hyperref}
\usepackage{tikz}
\usepackage{graphicx}
\usepackage{marginnote}
\usepackage{subcaption}
\usepackage{textcomp}
\usepackage{csvsimple}
\usepackage{booktabs}
\usepackage[utopia]{mathdesign}
%\usepackage{color}
%\renewcommand\UrlFont{\color{blue}\rmfamily}
%\usepackage[pagebackref=true,breaklinks=true,colorlinks,bookmarks=false]{hyperref}
%Viale dell'Universit� 16 - 35020 Legnaro (PD)
\begin{document}
%
\title{Is Segment Anything Model a revolution in precision agriculture?}
%
%\titlerunning{Abbreviated paper title}
% If the paper title is too long for the running head, you can set
% an abbreviated paper title here
%

% ARTICOLI DI SPUNTO
% https://www.sciencedirect.com/science/article/pii/S2214317323000112
%Prendo spunto da https://arxiv.org/pdf/2304.12306.pdf

\author{
Alberto Carraro\inst{1}
\orcidlink{0000-0002-9747-0978}
\and
Francesco Marinello\inst{1}
%\orcidID{2222--3333-4444-5555}
}
\authorrunning{A. Carraro and F. Marinello}
% First names are abbreviated in the running head.
% If there are more than two authors, 'et al.' is used.
%
\institute{TESAF Department, University of Padova}
%
\maketitle              % typeset the header of the contribution
%
\begin{abstract}
Precision agriculture uses accurate identification and mapping of crop features by automated mechanisms.  The use of computer vision techniques implemented by supervised deep learning systems to solve many precision agricultural problems necessitates large-scale data collection and prolonged ground truth annotation by humans. The so-called foundation models in artificial intelligence are therefore becoming more and more significant. Meta is working on a project called Segment Anything to provide a base model for image segmentation. Without extra training, it can accomplish zero-shot generalization to strange objects and images. This study evaluates the performance of Meta's Segment Anything Model (SAM) for the problem of semantic segmentation of objects in the context of precision agriculture.
\keywords{Precision agriculture \and 
Meta's Segment Anything Model (SAM) \and
Image segmentation \and
Computer vision}
\end{abstract}

\section{Introduction}

{
\color{red} The main points that the reader should get from this paper are:
\begin{itemize}
\item There are large labelled datasets for computer vision tasks but they contain images of easily recognizable subjects (animals, cars, roads, etc.)
\item Scientists can take open source architectures pre-trained on those large datasets but then they need to adapt them to their specific case, with additional ground-truth labelling (that requires experts) and additional training
\item SAM requires no additional training and the authors claim it has learned the very concept of "object", disregarding any specific semantic context: how well can it perform on non-ordinary images?
\end{itemize}
Bisogna prendere spunto da 
\begin{itemize}
\item \url{https://www.sciencedirect.com/science/article/pii/S2214317323000112}
\item \url{https://www.sciencedirect.com/science/article/pii/S0168169921002593/pdf}
\end{itemize}
}

Precision agriculture is a broad field that aims to improve agricultural practices' efficiency, productivity, and sustainability through science and technology. One of the critical challenges in precision agriculture is accurately identifying and mapping crop features and conditions such as diseases, plant height, leaf area index, and crop growth stage. Computer vision-based approaches have driven innovation in the agricultural industry and are nowadays used in scenarios such as pest identification \cite{yuan_advanced_2022}, precision livestock farming \cite{QIAO2019104958}, and many more.  Machine vision technologies support farmers and industries, saving time and costs while reaching very good efficacy. 

In computer vision, there are four primary tasks that can be combined and handled in different ways. One is image categorization, which arranges images according to a limited number of classes. Another is object detection, which creates bounding boxes, or tiny rectangles, around particular items in photos that fall into specified categories. Semantic segmentation seeks to categorize areas inside the images that have been labeled at the pixel level based on texture, color, and spatial distribution. Instance Segmentation, which combines object recognition and semantic segmentation, recognizes the various instances presented in the image together with their borders at the pixel level.

Since early applications of semantic segmentation, this task has been implemented in various ways. Depending on the lighting conditions and the sharpness of edges, various techniques have been adopted, such as colour space conversion and combination of colour channels \cite{RIEHLE2020105201}. Other implementations use machine-learning-based classification techniques such as decision trees \cite{YANG2015149} and clustering \cite{ZHANG202282}. Deep Learning is a subset of Machine Learning, where the primary tools are Deep Neural Networks. The use of Deep Neural Networks for the solution of computer vision tasks reached enormous diffusion and success also due to the public availability of models like VGG \cite{simonyan2015vgg}, U-Net \cite{ronneberger2015unet}, SegNet \cite{badrinarayanan2016segnet}, DeepLab \cite{chen2017deeplab}. These are Deep Neural Nets of various kinds that have been trained over large datasets. Table~\ref{tab:models_datasets} summarizes the main deep learning models and the datasets they were trained on, together with the task they are meant to perform.

\begin{table}[h!]
\centering
\begin{tabular}{|p{2cm}|p{5cm}|p{4.5cm}|}
\hline
\makecell{\textbf{Model}}
 & 
\makecell{\textbf{Dataset used for training}}  
& 
\makecell{\textbf{Computer Vision Task}}  \\
\hline
AlexNet~\cite{krizhevsky2012alexnet} & ImageNet~\cite{deng2009imagenet} (14,197,122 images) & Image \mbox{Classification} \\
\hline
VGGNet~\cite{simonyan2015vgg}  & ImageNet & Semantic Image \mbox{Segmentation} \\ 
\hline
ResNet~\cite{he2016resnet}  & ImageNet & Image \mbox{Classification}, Object \mbox{Detection} \\
\hline
U-Net~\cite{ronneberger2015unet} 
& 
\makecell[l]{
Various medical imaging datasets: \\
- MICCAI~\cite{landman2012miccai}~\cite{menze2015miccai}, \\
- ISIC~\cite{codella2019skin}
}
& 
Semantic Image \mbox{Segmentation} \\
\hline
DeepLab~\cite{chen2017deeplab} 
&
\makecell[l]{
- PASCAL VOC~\cite{everingham2010pascal},\\
- Cityscapes~\cite{cordts2016cityscapes} (5,000 images),\\
- ADE20K~\cite{zhou2017scene} (20,210 images),\\
- COCO~\cite{lin2014microsoft} (118,000 images) 
}
& 
Semantic Image \mbox{Segmentation} \\
\hline
YOLO~\cite{redmon2016yolo}  & COCO & Real-Time Object \mbox{Detection} \\
\hline
SegNet~\cite{badrinarayanan2016segnet} 
& 
\makecell[l]{
- CamVid~\cite{brostow2009semantic},\\
- Cityscapes, \\
- SUN RGB-D~\cite{song2015sun} (10,335 images),\\
- ADE20K
}
& 
Semantic Image \mbox{Segmentation} \\
\hline
EfficientNet~\cite{EfficientNet}
&
ImageNet
&
Semantic Image \mbox{Segmentation} \\
\hline
\end{tabular}
\caption{Most popular open source deep learning models and datasets used in computer vision tasks.}
\label{tab:models_datasets}
\end{table}

Open-source deep learning models have revolutionized the field of artificial intelligence by providing accessible and adaptable solutions for various tasks, allowing researchers and developers to utilize them for their own projects. A particularly successful principle in precision agriculture is that of Transfer Learning~\cite{zhuang2020comprehensive}, offering significant advantages in developing robust and accurate models for various agricultural tasks. By leveraging pre-trained deep learning models on large-scale general image datasets, transfer learning enables the transfer of knowledge from the source domain to the target agricultural domain. This approach allows researchers to overcome challenges associated with limited annotated agricultural datasets and improve model performance by initializing the network with learned features. Several studies have demonstrated the effectiveness of transfer learning in precision agriculture tasks, such as crop disease detection~\cite{paymode_transfer_2022}, crop type mapping~\cite{nowakowski_crop_2021}, and yield prediction~\cite{10.1145/3209811.3212707}. Transfer learning not only accelerates model training but also enhances the generalization and adaptability of the models to different agricultural environments, ultimately contributing to improved crop management and increased agricultural productivity.
 
Transfer learning still requires some ground-truth labelling and additional training in order to specialize the chosen Deep Learning model for the domain at hand. For this last purpose there are dataset specialised on the agricultural domain like the PlantVillage~\cite{mohanty2016plantvillage} project containing 54,306 images of 14 crop species with 26 diseases (or healthy) made openly available.

There are also large-scale datasets commonly used for semantic segmentation in precision agriculture. Some of the largest and widely used datasets are:

\begin{enumerate}
\item Plant Phenotyping Datasets: These datasets focus on crop and plant analysis, providing pixel-level annotations for various plant structures.
\begin{itemize}
\item PlantCLEF~\cite{plantclef}: A collection of plant images with annotations for leaf, stem, and flower structures.
\item Plant Phenotyping Dataset~\cite{plant_phenotyping_dataset}: Contains images of Arabidopsis plants with annotations for leaves and other plant components.
\end{itemize}
\item Crop Field Datasets: These datasets consist of aerial or satellite images of crop fields with annotations for different crops or objects of interest.
\begin{itemize}
\item CropSeg~\cite{cropseg}: A dataset with high-resolution aerial images of different crops, annotated at the pixel level.
\item Crop DeepLab Dataset~\cite{crop_deeplab_dataset}: Contains crop field images with fine-grained annotations for different crop types and objects.
\end{itemize}
\item Aerial Imagery Datasets: These datasets focus on aerial images captured by drones or satellites, providing annotations for various objects or classes of interest.
\begin{itemize}
\item ISPRS 2D Semantic Labeling~\cite{isprs_semantic_labeling}: A large-scale dataset with high-resolution aerial images and annotations for building, road, and vegetation classes.
\item DeepGlobe Land Cover Classification~\cite{deepglobe_land_cover}: Consists of satellite imagery with annotations for land cover classes, including agricultural areas.
\end{itemize}
\end{enumerate}

These datasets offer substantial amounts of labeled data for semantic segmentation tasks in precision agriculture. Researchers often utilize them to develop and evaluate models for crop analysis, disease detection, weed detection, and other applications in precision agriculture.

Recently, segmentation foundation models have seen tremendous advancements in the field of natural image segmentation~\cite{2023-SegGPT}\cite{2023-SEEM}, enabling accurate and efficient segmentation of objects in a fully automatic or interactive way. These models are typically based on transformer architectures and leverage pre-trained weights to achieve state-of-the-art performance and unprecedented generalisation ability on a wide range of natural images. Among these the Segment Anything project \cite{2023-SAM-Meta} by Meta\textsuperscript{\tiny\textregistered} is a task, a model, and a dataset for image segmentation. The Segment Aanything Model (SAM) in particular has learned a general notion of what objects are and this understanding enables \emph{zero-shot} generalization to unfamiliar objects and images without requiring additional training. SAM has been trained on the SA-1B dataset~\cite{SA-1B_dataset}, the largest segmentation dataset to date, with over 1 billion masks on 11 million images. 

SAM is designed and trained to be promptable, so its segmentation capabilities can be extended and transferred to new image distributions and tasks. We evaluate its capabilities on the specific tasks of segmenting objects 
inside the pictures of weed and crop of the CWFID dataset~\cite{CWFID_dataset}. 
%inside the pictures of vines of the ESCA dataset~\cite{ESCA_dataset}. 

\section{Materials and methods}

Our goal is to compare the performances of the foundational segmentation model SAM with no training. The chosen task is semantic segmentation i.e. to classify the pixels of each image into two different semantic categories: foliage (foreground) and background. This task could be an important first part in a Machine Learing pipeline whose later stages focus on the analysis of the Region of Interest inside the pictures, which in this case we considered to be the area occupied by foliage. Figure~\ref{fig:fore_vs_back} illustrates {\color{red} what we considered as foreground: all entities that were at the distance of within 1m from the camera.}

\begin{figure}[h!]
\centering
\includegraphics[width=0.3\textwidth]{imgs/fore_vs_back.png}
\caption{Foreground vs background.}
\label{fig:fore_vs_back}
\end{figure}

\subsection{Segment Anything}

The Segment Anything Model (SAM) utilises a transformer-based architecture~\cite{attention-Nips17}, which has been shown to be highly effective in natural language processing~\cite{GPT-3} and image recognition tasks~\cite{ViT2020}. Specifically, SAM uses a vision transformer-based \textbf{image encoder} to extract image features and \textbf{prompt encoders} to incorporate user interactions, followed by a \textbf{mask decoder} to generate segmentation results and confidence scores based on the image embedding, prompt embedding, and output token. 

\begin{figure}[h!]
\centering
\includegraphics[scale=0.5,width=0.95\textwidth]{imgs/SAM_model_diagram.png}
\caption{Segment Anything Model (SAM) overview. A heavyweight image encoder outputs an image embedding that can
then be efficiently queried by a variety of input prompts to produce object masks.}
\label{fig:sam-diagram}
\end{figure}

The prompt encoders are tailored for different user inputs. SAM supports four different prompts: points, boxes, texts, and masks. Each point is encoded by Fourier positional encoding~\cite{FourierPE-Nips20} and two learnable tokens for specifying foreground and background, respectively. The bounding box is encoded by the point encoding of its top-left corner and bottom-right corner. The free-form text is encoded by the pre-trained text-encoder in CLIP~\cite{CLIP-ICML2021}. The mask prompt has the same spatial resolution as the input image, which is encoded by convolution feature maps. 
Finally, the mask decoder employs a lightweight design, which consists of two transformer layers with a dynamic mask prediction head and an Intersection-over-Union (IoU) score regression head. The mask prediction head can generate three 4$\times$ downscaled masks, which correspond to the whole object, part, and subpart of the object, respectively.

\begin{figure}[h!]
\centering
\begin{subfigure}[b]{.45\linewidth}
    \includegraphics[width=\linewidth]{imgs/segmentation_modes_original.png}
    \caption{Original image}
\end{subfigure}
\begin{subfigure}[b]{.45\linewidth}
    \includegraphics[width=\linewidth]{imgs/segmentation_modes_everything.png}
    \caption{Fully automatic mode}
\end{subfigure}

\begin{subfigure}[b]{.45\linewidth}
    \includegraphics[width=\linewidth]{imgs/segmentation_modes_bbox.png}
        \caption{Bounding box mode}
\end{subfigure}
\begin{subfigure}[b]{.45\linewidth}
   \includegraphics[width=\linewidth]{imgs/segmentation_modes_1point.png}
   \caption{Additional point modality}
\end{subfigure}
%\begin{subfigure}[b]{.33\linewidth}
%    \includegraphics[width=\linewidth]{imgs/healthy_033_cam3_mask_over_healthy_033_cam3.png}
%\end{subfigure}
\caption{Segmentation results of SAM based on different segmentation modes.}
\label{fig:sam-modes}
\end{figure}

To summarize, when applying SAM for agricultural image segmentation, the segment-everything mode is prone to generate useless region partitions and the point-based mode is ambiguous and requires multiple prediction-correction iterations. In contrast, the bounding box-based mode can clearly specify the ROI and obtain reasonable segmentation results without multiple trials and errors. We argue that the bounding box-based segmentation mode has wider practical values than the segment-everything and point-based mode when using SAM in medical image segmentation tasks.

Since the image encoder can be applied prior to prompting the model, we can pre-compute the image embedding for all training images to avoid replicated computing of the image embedding per prompt, which can significantly improve the training efficiency. The mask decoder only needs to generate one mask rather than three masks because the bounding box prompt can clearly specify the expected segmentation target in most situations. 

\section{Experiments and Results}

The CWFID dataset~\cite{CWFID_dataset} contains 60 RGB images in png format taken at a working distance of approximately 30 cm from plants .....

Deep neural networks require large amounts of training data to tune millions of parameters and develop a learned model for subsequent predictions. While it is possible to crowdsource image annotation for natural scenes and collect large datsets as shown in Table~\ref{tab:models_datasets}, it is difficult to find experts to annotate images for specific scientific domains. Each research topic requires researchers to annotate their own images. Annotation being a very time consuming task, a researcher can end up with a handful of annotated images containing only tens of labeled objects. 

Our goal was to evalluate the performances of the foundational segmentation model SAM with no training. The chosen task is semantic segmentation i.e. to classify the pixels of each image into two different semantic categories: plants and background. This task could be an important first part in a Machine Learing pipeline whose later stages focus on the analysis of the Region of Interest inside the pictures, which in this case we considered to be the area occupied by plants.

\begin{figure}[h!]
\centering
\begin{subfigure}[b]{.47\linewidth}
\includegraphics[width=\linewidth]{imgs/050_image.png}
\caption{}
\label{fig:sam1a}
\end{subfigure}
\begin{subfigure}[b]{.47\linewidth}
\includegraphics[width=\linewidth]{imgs/050_ground_truth_mask.png}
\caption{}
\label{fig:sam1b}
\end{subfigure}
\caption{Ground truth segmentation of foliage (hand-made mask) for the file 050\_image.png}
\label{fig:true_segmentation}
\end{figure}

The 60 image dataset was captured at a commercial organic carrot farm in Northern Germany in 2013 just before manual weed control was applied. The carrots were grown in single rows on small soil dams.

The 60 images have been manually annotated by the authors of the CWFID dataset ...... The result of the manual segmentation are a total of 60 ``.yaml" files containing the coordinates of contour points that define masks delineating weed and crop inside each image.....

\begin{table}[h!]
\centering
\begin{tabular}{|p{3.0cm}|p{3.5cm}|p{3.5cm}|}
\hline
\makecell{\textbf{Image name}} 
& 
\makecell{\textbf{Number of points} \\ \textbf{on the border}}
& 
\makecell{\textbf{Manual annotation time} \\ \textbf{(seconds)}} \\
\hline
001\_annotation.yaml & 233 & 0 \\
002\_annotation.yaml & 106 & 0 \\
003\_annotation.yaml & 54 & 0 \\
004\_annotation.yaml & 124 & 0 \\
005\_annotation.yaml & 41 & 0 \\
006\_annotation.yaml & 95 & 0 \\
007\_annotation.yaml & 217 & 0 \\
008\_annotation.yaml & 108 & 0 \\
009\_annotation.yaml & 127 & 0 \\
010\_annotation.yaml & 106 & 0 \\
%011\_annotation.yaml & 98 & 0 \\
%012\_annotation.yaml & 39 & 0 \\
%013\_annotation.yaml & 118 & 0 \\
%014\_annotation.yaml & 47 & 0 \\
%015\_annotation.yaml & 45 & 0 \\
%016\_annotation.yaml & 167 & 0 \\
%017\_annotation.yaml & 98 & 0 \\
%018\_annotation.yaml & 101 & 0 \\
%019\_annotation.yaml & 68 & 0 \\
%020\_annotation.yaml & 33 & 0 \\
%021\_annotation.yaml & 49 & 0 \\
%022\_annotation.yaml & 93 & 0 \\
%023\_annotation.yaml & 79 & 0 \\
%024\_annotation.yaml & 60 & 0 \\
%025\_annotation.yaml & 78 & 0 \\
%026\_annotation.yaml & 50 & 0 \\
%027\_annotation.yaml & 90 & 0 \\
%028\_annotation.yaml & 163 & 0 \\
%029\_annotation.yaml & 137 & 0 \\
%030\_annotation.yaml & 33 & 0 \\
%031\_annotation.yaml & 59 & 0 \\
%032\_annotation.yaml & 69 & 0 \\
%033\_annotation.yaml & 110 & 0 \\
%034\_annotation.yaml & 122 & 0 \\
%035\_annotation.yaml & 116 & 0 \\
%036\_annotation.yaml & 189 & 0 \\
%037\_annotation.yaml & 66 & 0 \\
%038\_annotation.yaml & 139 & 0 \\
%039\_annotation.yaml & 79 & 0 \\
%040\_annotation.yaml & 91 & 0 \\
%041\_annotation.yaml & 113 & 0 \\
%042\_annotation.yaml & 50 & 0 \\
%043\_annotation.yaml & 49 & 0 \\
%044\_annotation.yaml & 41 & 0 \\
%045\_annotation.yaml & 131 & 0 \\
%046\_annotation.yaml & 29 & 0 \\
%047\_annotation.yaml & 44 & 0 \\
%048\_annotation.yaml & 59 & 0 \\
%049\_annotation.yaml & 38 & 0 \\
050\_annotation.yaml & 18 & 0 \\
051\_annotation.yaml & 150 & 0 \\
052\_annotation.yaml & 155 & 0 \\
053\_annotation.yaml & 55 & 0 \\
054\_annotation.yaml & 115 & 0 \\
055\_annotation.yaml & 105 & 0 \\
056\_annotation.yaml & 123 & 0 \\
057\_annotation.yaml & 148 & 0 \\
058\_annotation.yaml & 209 & 0 \\
059\_annotation.yaml & 204 & 0 \\
060\_annotation.yaml & 171 & 0 \\
\hline

& 
\makecell{\textbf{Sum}}
& 
\makecell{\textbf{Sum} } \\
\hline
 & 5904 & 0 \\
 \hline

& 
\makecell{\textbf{Average}}
& 
\makecell{\textbf{Average} } \\
\hline
 & 98.4 & 0 \\
\hline
\end{tabular}
\caption{Time spent for drawing the masks manually (ground-truth labelling).}
\label{tab:image_annotation_manual}
\end{table}

In Table~\ref{tab:image_annotation_on_manual} we recorded the time used to manually draw precise segmentation masks for 60 images. We keep the so-obtained 60 masks as a reference for the rest of the paper and consider them as ground-truth to measure performance of the two segmentation tools analyzed. In~\ref{tab:definitions} we summarized the classification of pixel predictions in the context of semantic segmentation applied to the task of partitioning an image into its Region Of Interest (ROI) and its background.{\color{red} Frase contorta.}

\begin{table}[h!]
\centering
\begin{tabular}{|p{5.5cm}|p{5.5cm}|}
\hline
\makecell{\textbf{Correct predictions}}
&
\makecell{\textbf{Incorrect predictions}} \\
\hline

True Positive (TP):
the pixel has been predicted as part of the ROI and
actually belongs to the ROI 

 & 

False Positive (FP):
the pixel has been predicted as part of the ROI but 
does not belong to the ROI \\

\hline

True Positive (TN):
the pixel has been predicted as part of the background and
actually belongs to the background  

& 

True Positive (FN):
the pixel has been predicted as part of the background but
actually belongs to the ROI  \\

\hline
\end{tabular}
\caption{Definitions of TP, TN, FP, and FN}
\label{tab:definitions}
\end{table}

\begin{table}[h!]
\centering
\begin{tabular}{|p{3.0cm}|p{6.3cm}|p{4.0cm}|}
\hline
\makecell{\textbf{Measure name}}
&
\makecell{\textbf{Formula}} 
&
\makecell{\textbf{Description}} \\
\hline
Intersection over Union (IoU), or Jaccard Index
 & 
$$\frac{|M \cap G|}{|M \cup G|}$$
&
measures the overlap between the predicted segmentation mask and the ground truth mask. It is calculated as the ratio of the intersection area to the union area of the two masks
\\

\hline

Dice Coefficient (DC)

& 
$$\frac{2|M \cap G|}{|M| + |G|}$$
&
measures the similarity between the predicted boundaries and the ground truth boundaries. It calculates the ratio of twice the intersection of the boundaries to the sum of their areas
\\

\hline

Pixel Accuracy (PA)

& 
$$\frac{TP + TN}{TP + TN + FP + FN}$$
&
measures the percentage of correctly classified pixels compared to the total number of pixels in the image. It calculates the overall accuracy of the segmentation
\\

\hline

Precision (P)

& 
$$\frac{TP}{TP + FP} $$
&
measures the proportion of positive predictions relative to total positive predictions
\\

\hline

Recall (R)

& 
$$\frac{TP}{TP + FN}$$
& 
measures the proportion of correct positive predictions relative to total actual positives
\\
\hline
F1-score

& 
$$ \frac{2 \cdot P \cdot R}{P+R} $$
& 
combines precision and recall using their harmonic mean; meaningful when there is imbalance between positives and negtives in the dataset.
\\
\hline
Hausdorff Distance

& 

$$ \mathrm{max} \left\{ \mathrm{max}_{x \in \partial G} \mathrm{d}(x, \partial M), \mathrm{max}_{y \in \partial M} \mathrm{d}(y, \partial G) \right\}$$
&
measures the maximum distance (in millimeters) between the predicted contour and the ground truth contour. It quantifies the largest discrepancy between the two contours and is sensitive to outliers or large deviations
\\
\hline
Average Symmetric Surface Distance (ASSD)

& 
$$ \frac{\sum_{x \in \partial G} \mathrm{d}(x, \partial M) +  \sum_{y \in \partial M} \mathrm{d}(y, \partial G)}{|\partial G| + |\partial M|}$$
&
measures the average distance between the predicted contour and the ground truth contour. Unlike the Hausdorff distance, it takes into account both the under-segmentation (missing parts of the object) and over-segmentation (extraneous regions
\\
\hline
\end{tabular}
\caption{Definitions of each similarity / dissimilarity measure used in this paper.}
\label{tab:metrics_used}
\end{table}

\subsection{Segmentation of plant images with SAM}

SAM supports three main segmentation modes: segmenting everything in a fully automatic way, bounding box mode, and point mode. The segment-everything mode divides the whole image into six regions based on the image intensity (Fig.~\ref{fig:sam-modes}b).

We segmented {\color{red} 10} images with bounding box mode and other {\color{red} 10} images with point mode. These modes are semi-automatic, i.e. they require a human input for each image.

\begin{table}[h!]
\centering
\begin{tabular}{|p{3.0cm}|p{3.0cm}|p{3.0cm}|p{3.0cm}|}
\hline
\makecell{\textbf{Annotation file}}  
&
\makecell{\textbf{\# points inside mask}}
&
\makecell{\textbf{\# points outside mask}}
&
\makecell{\textbf{Annotation time} \\ \textbf{point mode} \\ \textbf{(seconds)}} \\
\hline
001\_annotation.json & 41 & 20 & 101.696 \\
002\_annotation.json & 35 & 21 & 95.565 \\
003\_annotation.json & 32 & 22 & 106.857 \\
004\_annotation.json & 46 & 29 & 125.596 \\
005\_annotation.json & 43 & 49 & 148.657 \\
006\_annotation.json & 53 & 71 & 210.004 \\
007\_annotation.json & 72 & 63 & 201.255 \\
008\_annotation.json & 58 & 68 & 186.949 \\
009\_annotation.json & 59 & 81 & 193.657 \\
010\_annotation.json & 51 & 68 & 197.857 \\
%011\_annotation.json & 82 & 84 & 224.888 \\
%012\_annotation.json & 31 & 77 & 124.534 \\
%013\_annotation.json & 60 & 73 & 186.33 \\
%014\_annotation.json & 44 & 64 & 145.737 \\
%015\_annotation.json & 49 & 81 & 146.716 \\
%016\_annotation.json & 79 & 76 & 209.166 \\
%017\_annotation.json & 53 & 78 & 151.232 \\
%018\_annotation.json & 60 & 89 & 155.673 \\
%019\_annotation.json & 63 & 85 & 162.488 \\
%020\_annotation.json & 30 & 112 & 145.248 \\
%021\_annotation.json & 39 & 123 & 151.843 \\
%022\_annotation.json & 75 & 112 & 242.476 \\
%023\_annotation.json & 52 & 116 & 196.753 \\
%024\_annotation.json & 69 & 93 & 270.024 \\
%025\_annotation.json & 76 & 82 & 223.944 \\
%026\_annotation.json & 32 & 96 & 153.957 \\
%027\_annotation.json & 82 & 117 & 292.686 \\
%028\_annotation.json & 80 & 112 & 264.466 \\
%029\_annotation.json & 120 & 95 & 288.516 \\
%030\_annotation.json & 24 & 119 & 122.353 \\
%031\_annotation.json & 54 & 89 & 199.722 \\
%032\_annotation.json & 70 & 118 & 261.031 \\
%033\_annotation.json & 65 & 121 & 257.678 \\
%034\_annotation.json & 69 & 99 & 293.182 \\
%035\_annotation.json & 62 & 121 & 244.333 \\
%036\_annotation.json & 70 & 93 & 219.324 \\
%037\_annotation.json & 47 & 130 & 206.17 \\
%038\_annotation.json & 74 & 123 & 239.44 \\
%039\_annotation.json & 46 & 125 & 222.824 \\
%040\_annotation.json & 82 & 135 & 385.179 \\
%041\_annotation.json & 45 & 77 & 176.355 \\
%042\_annotation.json & 29 & 79 & 109.598 \\
%043\_annotation.json & 30 & 79 & 131.012 \\
%044\_annotation.json & 33 & 120 & 120.934 \\
%045\_annotation.json & 65 & 103 & 158.763 \\
%046\_annotation.json & 19 & 117 & 105.836 \\
%047\_annotation.json & 29 & 145 & 117.797 \\
%048\_annotation.json & 45 & 137 & 159.498 \\
%049\_annotation.json & 44 & 114 & 175.778 \\
050\_annotation.json & 19 & 138 & 98.891 \\
051\_annotation.json & 118 & 176 & 388.167 \\
052\_annotation.json & 98 & 171 & 276.081 \\
053\_annotation.json & 32 & 140 & 114.474 \\
054\_annotation.json & 67 & 120 & 254.87 \\
055\_annotation.json & 75 & 138 & 337.862 \\
056\_annotation.json & 60 & 180 & 338.139 \\
057\_annotation.json & 77 & 167 & 293.152 \\
058\_annotation.json & 95 & 184 & 310.982 \\
059\_annotation.json & 79 & 166 & 338.914 \\
060\_annotation.json & 94 & 187 & 276.054 \\
\hline

& 
\makecell{\textbf{Sum}}
& 
\makecell{\textbf{Sum}}
& 
\makecell{\textbf{Sum}} \\
\hline
 & 3482 & 6268 & 12239.163 \\
 \hline

& 
\makecell{\textbf{Average}}
& 
\makecell{\textbf{Average}}
& 
\makecell{\textbf{Average}} \\
\hline
 & 58.033 & 104.467 & 203.986  \\
\hline
\end{tabular}
\caption{Time spent to place points for SAM predictor.}
\label{tab:image_annotation_on_SAM_predictor}
\end{table}

Figure~\ref{fig:sam-modes} shows the results of the three segmentation modes on an image of vines from the ESCA dataset~\cite{ESCA_dataset} and Table~\ref{tab:image_annotation_on_SAM_manual} reports the exact time necessary to manually draw segmentation masks for the 10 selected images~\footnote{The results are based on the online demo: \url{https://segment-anything.com/demo}.}.

We launched a batch segmentation job exploiting the fully-automatic mode on the entire set of 1770 images of the Esca dataset, using as hardware platform an NVIDIA RTX\textsuperscript{\texttrademark} A6000 GPU. For the instantiation of the model we adopted the {\tt ViT-H}~\cite{ViT-H} checkpoint. SAM produced for each segmented image a folder containing one mask for each object found in the image (between 100 and 200 objects). In order to process 888 images of the esca folder took 55 minutes and 13.8 seconds. In order to process 882 images of the healthy folder took 69 minutes and 19.4 seconds to complete the merge of masks took 48 minutes and 45.4 seconds to complete.

%%%%%%%%%%
% It does not make sense to record segmentation time of individual images.
% We just report global time  of segmentation for the entire dataset
%%%%%%%%%%
\begin{table}[h!]
\centering
\begin{tabular}{|p{3.0cm}|p{3.0cm}|p{3.5cm}|p{3.5cm}|}
\hline
 &  & \multicolumn{2}{c|}{\textbf{Image segmentation time (minutes)}} \\
\hline
\makecell{\textbf{Image folder}}
&
\makecell{\textbf{Number of images}}
&
\makecell{\textbf{Object segmentation} \\  \textbf{(automatic mode)}}
&
\makecell{\textbf{Object masks merging}} \\
\hline
Esca     & 888 & 55' and 13.8''   & ???  \\
\hline
Healthy & 882 &  69'' and 19.4'' & 48' and 45.4'' \\
\hline
\end{tabular}
\caption{Image segmentation time with SAM in automatic batch mode.}
\label{tab:image_segmentation_on_SAM_automatic}
\end{table}

%caThe project implies the public release three \emph{checkpoints} {\tt ViT-L} \cite{ViT-L}, {\tt ViT-B} \cite{ViT-B} and {\tt ViT-H} \cite{ViT-H}. Each checkpoint is a large file containing the weights for a deep neural network structure which has been obtained by training with different hyperparameters on the SA-1B dataset \cite{SA-1B_dataset}. The latter is the largest segmentation dataset to date, with over 1 billion masks on 11M licensed and privacy-respecting images. With the Segment Anything Model (SAM), the literature usually refers indistinctly to one of the three versions of the deep neural network created using one of the checkpoints. 

\begin{figure}[h!]
\centering
\begin{subfigure}[b]{.33\linewidth}
\includegraphics[width=\linewidth]{imgs/18_over_healthy_033_cam3.png}
\caption{}
\label{fig:sam1a}
\end{subfigure}
\begin{subfigure}[b]{.33\linewidth}
\includegraphics[width=\linewidth]{imgs/19_over_healthy_033_cam3.png}
\caption{}
\label{fig:sam1b}
\end{subfigure}%%%%%%%%%%%%%%%%%%%%%%%%%%%%%

%\begin{subfigure}[b]{.33\linewidth}
%\includegraphics[width=\linewidth]{imgs/23_over_healthy_033_cam3.png}
%\caption{}
%\label{fig:sam1c}
%\end{subfigure}
%\begin{subfigure}[b]{.33\linewidth}
%\includegraphics[width=\linewidth]{imgs/20_over_healthy_033_cam3.png}
%\caption{}
%\label{fig:sam1d}
%\end{subfigure}%%%%%%%%%%%%%%%%%%%%%%%%%%%%%

\begin{subfigure}[b]{.66\linewidth}
\includegraphics[width=\linewidth]{imgs/healthy_033_cam3_mask_over_healthy_033_cam3.png}
\caption{}
\label{fig:sam1e}
\end{subfigure}
\label{fig:individual_SAM_mask_and_their_union}
\caption{Two masks produced by SAM in automatic mode on {\color{red} healthy1.jpg} and the union of all masks produced for the same image.}
\end{figure}

\begin{table}[h!]
\centering
\begin{tabular}{|p{2.5cm}|p{1.0cm}|p{1.0cm}|p{1.0cm}|p{1.0cm}|p{1.0cm}|p{1.0cm}|p{1.0cm}|p{1.0cm}|}
\hline
\makecell{\textbf{Image name}} 
&
\makecell{\textbf{IoU}} 
&
 \makecell{\textbf{DC}}
  &
 \makecell{\textbf{PA}} 
 & 
 \makecell{\textbf{Prec}} 
 &
 \makecell{\textbf{Rec}} 
 &
  \makecell{\textbf{F1}} 
 &
  \makecell{\textbf{ASSD} \\ \textbf{(mm)}} 
   &
  \makecell{\textbf{HD} \\ \textbf{(mm)}} 
  \\
\hline
001\_image.png & 0.804 & 0.892 & 0.962 & 0.157 & 0.887 & 0.266 & 6.456 & 328.868 \\
002\_image.png & 0.779 & 0.876 & 0.971 & 0.101 & 0.89 & 0.182 & 5.887 & 262.574 \\
003\_image.png & 0.829 & 0.907 & 0.985 & 0.072 & 0.872 & 0.134 & 6.274 & 329.474 \\
004\_image.png & 0.804 & 0.892 & 0.976 & 0.098 & 0.857 & 0.175 & 2.832 & 147.872 \\
005\_image.png & 0.861 & 0.926 & 0.991 & 0.054 & 0.931 & 0.102 & 3.275 & 237.306 \\
006\_image.png & 0.769 & 0.87 & 0.976 & 0.081 & 0.806 & 0.147 & 2.999 & 127.44 \\
007\_image.png & 0.819 & 0.9 & 0.968 & 0.145 & 0.896 & 0.25 & 2.595 & 170.88 \\
008\_image.png & 0.81 & 0.895 & 0.979 & 0.089 & 0.923 & 0.162 & 3.451 & 164.657 \\
009\_image.png & 0.815 & 0.898 & 0.981 & 0.081 & 0.933 & 0.15 & 5.787 & 361.334 \\
010\_image.png & 0.792 & 0.884 & 0.983 & 0.064 & 0.939 & 0.12 & 5.45 & 311.045 \\
%011\_image.png & 0.827 & 0.905 & 0.986 & 0.068 & 0.935 & 0.127 & 3.918 & 240.075 \\
%012\_image.png & 0.759 & 0.863 & 0.991 & 0.029 & 0.937 & 0.057 & 18.426 & 320.339 \\
%013\_image.png & 0.858 & 0.923 & 0.988 & 0.073 & 0.92 & 0.135 & 4.568 & 314.428 \\
%014\_image.png & 0.866 & 0.928 & 0.994 & 0.041 & 0.927 & 0.079 & 8.499 & 310.645 \\
%015\_image.png & 0.832 & 0.908 & 0.988 & 0.058 & 0.948 & 0.109 & 14.337 & 477.009 \\
%016\_image.png & 0.837 & 0.911 & 0.984 & 0.081 & 0.924 & 0.148 & 2.715 & 229.682 \\
%017\_image.png & 0.876 & 0.934 & 0.993 & 0.048 & 0.934 & 0.091 & 3.288 & 174.27 \\
%018\_image.png & 0.884 & 0.938 & 0.99 & 0.074 & 0.925 & 0.137 & 7.044 & 360.555 \\
%019\_image.png & 0.833 & 0.909 & 0.992 & 0.04 & 0.955 & 0.077 & 3.977 & 317.918 \\
%020\_image.png & 0.725 & 0.841 & 0.993 & 0.017 & 0.979 & 0.034 & 29.632 & 279.852 \\
%021\_image.png & 0.595 & 0.746 & 0.985 & 0.022 & 0.958 & 0.043 & 66.132 & 583.545 \\
%022\_image.png & 0.806 & 0.892 & 0.987 & 0.055 & 0.934 & 0.104 & 11.931 & 587.892 \\
%023\_image.png & 0.826 & 0.905 & 0.988 & 0.059 & 0.87 & 0.111 & 8.515 & 366.072 \\
%024\_image.png & 0.771 & 0.871 & 0.983 & 0.056 & 0.945 & 0.105 & 13.82 & 316.792 \\
%025\_image.png & 0.818 & 0.9 & 0.986 & 0.063 & 0.927 & 0.118 & 9.818 & 258.699 \\
%026\_image.png & 0.812 & 0.896 & 0.99 & 0.045 & 0.91 & 0.086 & 15.075 & 294.715 \\
%027\_image.png & 0.771 & 0.871 & 0.979 & 0.07 & 0.952 & 0.13 & 9.971 & 288.647 \\
%028\_image.png & 0.813 & 0.897 & 0.978 & 0.097 & 0.878 & 0.175 & 4.088 & 218.387 \\
%029\_image.png & 0.832 & 0.908 & 0.965 & 0.172 & 0.898 & 0.289 & 3.807 & 224.27 \\
%030\_image.png & 0.807 & 0.893 & 0.995 & 0.021 & 0.924 & 0.041 & 21.349 & 744.614 \\
%031\_image.png & 0.9 & 0.948 & 0.986 & 0.13 & 0.932 & 0.228 & 2.076 & 71.176 \\
%032\_image.png & 0.769 & 0.87 & 0.974 & 0.086 & 0.883 & 0.157 & 2.481 & 130.015 \\
%033\_image.png & 0.872 & 0.932 & 0.988 & 0.08 & 0.919 & 0.148 & 2.164 & 32.45 \\
%034\_image.png & 0.674 & 0.805 & 0.952 & 0.098 & 0.915 & 0.178 & 28.559 & 315.691 \\
%035\_image.png & 0.831 & 0.908 & 0.985 & 0.073 & 0.959 & 0.135 & 5.341 & 196.759 \\
%036\_image.png & 0.864 & 0.927 & 0.982 & 0.114 & 0.918 & 0.203 & 2.974 & 149.496 \\
%037\_image.png & 0.789 & 0.882 & 0.99 & 0.038 & 0.957 & 0.073 & 15.494 & 484.846 \\
%038\_image.png & 0.847 & 0.917 & 0.979 & 0.114 & 0.904 & 0.202 & 3.387 & 188.452 \\
%039\_image.png & 0.728 & 0.843 & 0.988 & 0.033 & 0.958 & 0.063 & 30.562 & 365.472 \\
%040\_image.png & 0.834 & 0.909 & 0.978 & 0.111 & 0.927 & 0.199 & 4.479 & 331.157 \\
%041\_image.png & 0.835 & 0.91 & 0.99 & 0.05 & 0.938 & 0.096 & 5.494 & 425.169 \\
%042\_image.png & 0.836 & 0.911 & 0.992 & 0.042 & 0.947 & 0.08 & 23.291 & 323.858 \\
%043\_image.png & 0.846 & 0.917 & 0.995 & 0.03 & 0.959 & 0.058 & 6.789 & 245.367 \\
%044\_image.png & 0.739 & 0.85 & 0.99 & 0.029 & 0.958 & 0.056 & 28.751 & 444.011 \\
%045\_image.png & 0.804 & 0.891 & 0.983 & 0.071 & 0.94 & 0.132 & 16.478 & 373.289 \\
%046\_image.png & 0.248 & 0.397 & 0.93 & 0.023 & 0.956 & 0.045 & 129.762 & 491.304 \\
%047\_image.png & 0.793 & 0.885 & 0.992 & 0.031 & 0.947 & 0.059 & 34.63 & 344.501 \\
%048\_image.png & 0.791 & 0.883 & 0.99 & 0.038 & 0.931 & 0.073 & 17.585 & 351.552 \\
%049\_image.png & 0.831 & 0.908 & 0.993 & 0.036 & 0.962 & 0.07 & 10.07 & 321.885 \\
050\_image.png & 0.276 & 0.433 & 0.966 & 0.013 & 0.805 & 0.025 & 150.533 & 588.205 \\
051\_image.png & 0.846 & 0.916 & 0.975 & 0.138 & 0.912 & 0.24 & 6.189 & 499.401 \\
052\_image.png & 0.803 & 0.89 & 0.974 & 0.106 & 0.928 & 0.19 & 11.023 & 485.956 \\
053\_image.png & 0.722 & 0.839 & 0.978 & 0.056 & 0.944 & 0.106 & 32.785 & 456.054 \\
054\_image.png & 0.89 & 0.942 & 0.988 & 0.099 & 0.919 & 0.178 & 2.697 & 469.998 \\
055\_image.png & 0.802 & 0.89 & 0.979 & 0.085 & 0.884 & 0.155 & 11.072 & 358.236 \\
056\_image.png & 0.789 & 0.882 & 0.972 & 0.103 & 0.882 & 0.184 & 7.121 & 238.481 \\
057\_image.png & 0.81 & 0.895 & 0.977 & 0.096 & 0.905 & 0.174 & 6.389 & 218.009 \\
058\_image.png & 0.867 & 0.929 & 0.98 & 0.132 & 0.9 & 0.231 & 4.626 & 173.807 \\
059\_image.png & 0.756 & 0.861 & 0.967 & 0.101 & 0.88 & 0.181 & 7.385 & 252.634 \\
060\_image.png & 0.858 & 0.924 & 0.983 & 0.102 & 0.924 & 0.184 & 5.307 & 319.265 \\
 \hline

& 
\makecell{\textbf{Avg.}}
& 
\makecell{\textbf{Avg.}}
& 
\makecell{\textbf{Avg.}} 
& 
\makecell{\textbf{Avg.}}
& 
\makecell{\textbf{Avg.}}
& 
\makecell{\textbf{Avg.}}
& 
\makecell{\textbf{Avg.}}
& 
\makecell{\textbf{Avg.}} \\
\hline
 & ..... & ..... & ..... & ..... & ..... & .....  & .....  & .....  \\
\hline
\end{tabular}
\caption{Evaluation of semantic segmentation performed by SAM in predictor mode with input points.}
\label{tab:segmentation_evaluation_SAM_predictor}
\end{table}

\begin{table}[h!]
\centering
\begin{tabular}{|p{2.5cm}|p{1.0cm}|p{1.0cm}|p{1.0cm}|p{1.0cm}|p{1.0cm}|p{1.0cm}|p{1.0cm}|p{1.0cm}|}
\hline
\makecell{\textbf{Image name}} 
&
\makecell{\textbf{IoU}} 
&
 \makecell{\textbf{DC}}
  &
 \makecell{\textbf{PA}} 
 & 
 \makecell{\textbf{Prec}} 
 &
 \makecell{\textbf{Rec}} 
 &
  \makecell{\textbf{F1}} 
 &
  \makecell{\textbf{ASSD} \\ \textbf{(mm)}} 
   &
  \makecell{\textbf{HD} \\ \textbf{(mm)}} 
  \\
\hline
001\_image.png & 0.285 & 0.444 & 0.744 & 0.102 & 0.579 & 0.174 & 29.965 & 384.033 \\
002\_image.png & 0.167 & 0.286 & 0.666 & 0.067 & 0.588 & 0.12 & 38.636 & 274.39 \\
003\_image.png & 0.234 & 0.379 & 0.757 & 0.074 & 0.891 & 0.137 & 95.89 & 638.719 \\
004\_image.png & 0.186 & 0.314 & 0.716 & 0.065 & 0.571 & 0.117 & 41.91 & 326.686 \\
005\_image.png & 0.13 & 0.23 & 0.694 & 0.046 & 0.786 & 0.086 & 120.494 & 597.759 \\
006\_image.png & 0.168 & 0.287 & 0.721 & 0.056 & 0.56 & 0.102 & 77.322 & 518.383 \\
007\_image.png & 0.169 & 0.29 & 0.725 & 0.056 & 0.346 & 0.097 & 50.349 & 407.836 \\
008\_image.png & 0.11 & 0.198 & 0.729 & 0.034 & 0.35 & 0.061 & 55.287 & 399.606 \\
009\_image.png & 0.129 & 0.229 & 0.724 & 0.041 & 0.47 & 0.075 & 79.684 & 612.148 \\
010\_image.png & 0.046 & 0.088 & 0.67 & 0.016 & 0.232 & 0.03 & 112.531 & 615.325 \\
%011\_image.png & 0.132 & 0.234 & 0.736 & 0.04 & 0.552 & 0.075 & 47.787 & 450.245 \\
%012\_image.png & 0.071 & 0.133 & 0.703 & 0.023 & 0.728 & 0.044 & 110.276 & 616.41 \\
%013\_image.png & 0.18 & 0.305 & 0.744 & 0.056 & 0.71 & 0.104 & 69.631 & 481.087 \\
%014\_image.png & 0.094 & 0.171 & 0.677 & 0.033 & 0.753 & 0.064 & 139.856 & 780.651 \\
%015\_image.png & 0.164 & 0.282 & 0.737 & 0.052 & 0.85 & 0.097 & 117.336 & 601.748 \\
%016\_image.png & 0.195 & 0.326 & 0.756 & 0.059 & 0.677 & 0.109 & 109.837 & 628.5 \\
%017\_image.png & 0.147 & 0.256 & 0.718 & 0.048 & 0.942 & 0.092 & 94.432 & 498.206 \\
%018\_image.png & 0.269 & 0.424 & 0.812 & 0.069 & 0.861 & 0.128 & 75.694 & 513.522 \\
%019\_image.png & 0.118 & 0.212 & 0.702 & 0.04 & 0.957 & 0.077 & 95.412 & 458.705 \\
%020\_image.png & 0.066 & 0.124 & 0.755 & 0.017 & 0.975 & 0.034 & 146.913 & 517.92 \\
%021\_image.png & 0.035 & 0.067 & 0.687 & 0.011 & 0.492 & 0.022 & 242.242 & 774.213 \\
%022\_image.png & 0.02 & 0.04 & 0.058 & 0.02 & 0.334 & 0.037 & 22.044 & 472.488 \\
%023\_image.png & 0.055 & 0.105 & 0.082 & 0.054 & 0.789 & 0.1 & 18.808 & 622.637 \\
%024\_image.png & 0.053 & 0.101 & 0.461 & 0.03 & 0.514 & 0.057 & 73.194 & 524.681 \\
%025\_image.png & 0.106 & 0.192 & 0.684 & 0.037 & 0.553 & 0.07 & 88.206 & 446.077 \\
%026\_image.png & 0.13 & 0.23 & 0.772 & 0.034 & 0.69 & 0.065 & 120.902 & 613.998 \\
%027\_image.png & 0.153 & 0.265 & 0.777 & 0.04 & 0.548 & 0.075 & 55.219 & 317.002 \\
%028\_image.png & 0.103 & 0.187 & 0.7 & 0.035 & 0.313 & 0.062 & 90.68 & 676.244 \\
%029\_image.png & 0.155 & 0.268 & 0.66 & 0.062 & 0.324 & 0.104 & 45.847 & 418.779 \\
%030\_image.png & 0.055 & 0.105 & 0.654 & 0.02 & 0.892 & 0.04 & 296.39 & 964.718 \\
%031\_image.png & 0.284 & 0.442 & 0.731 & 0.106 & 0.765 & 0.187 & 67.482 & 452.623 \\
%032\_image.png & 0.157 & 0.271 & 0.737 & 0.049 & 0.5 & 0.089 & 70.858 & 619.604 \\
%033\_image.png & 0.179 & 0.303 & 0.725 & 0.06 & 0.686 & 0.11 & 100.46 & 584.014 \\
%034\_image.png & 0.103 & 0.187 & 0.66 & 0.039 & 0.363 & 0.071 & 88.405 & 551.943 \\
%035\_image.png & 0.226 & 0.369 & 0.801 & 0.058 & 0.764 & 0.108 & 83.423 & 565.468 \\
%036\_image.png & 0.221 & 0.362 & 0.697 & 0.086 & 0.689 & 0.153 & 46.349 & 312.181 \\
%037\_image.png & 0.061 & 0.115 & 0.709 & 0.019 & 0.477 & 0.036 & 175.942 & 687.284 \\
%038\_image.png & 0.102 & 0.185 & 0.468 & 0.06 & 0.48 & 0.107 & 61.447 & 539.281 \\
%039\_image.png & 0.082 & 0.152 & 0.723 & 0.025 & 0.729 & 0.048 & 111.03 & 413.192 \\
%040\_image.png & 0.188 & 0.317 & 0.772 & 0.053 & 0.44 & 0.094 & 71.271 & 416.544 \\
%041\_image.png & 0.111 & 0.2 & 0.773 & 0.028 & 0.528 & 0.054 & 106.818 & 612.733 \\
%042\_image.png & 0.109 & 0.197 & 0.656 & 0.042 & 0.961 & 0.081 & 77.454 & 450.04 \\
%043\_image.png & 0.109 & 0.196 & 0.78 & 0.027 & 0.858 & 0.052 & 96.475 & 497.213 \\
%044\_image.png & 0.078 & 0.145 & 0.657 & 0.029 & 0.965 & 0.057 & 157.657 & 655.857 \\
%045\_image.png & 0.18 & 0.305 & 0.772 & 0.05 & 0.664 & 0.093 & 102.817 & 489.434 \\
%046\_image.png & 0.044 & 0.084 & 0.487 & 0.023 & 0.964 & 0.046 & 162.531 & 658.238 \\
%047\_image.png & 0.116 & 0.208 & 0.765 & 0.031 & 0.954 & 0.06 & 154.174 & 664.07 \\
%048\_image.png & 0.075 & 0.139 & 0.679 & 0.026 & 0.632 & 0.05 & 106.238 & 618.044 \\
%049\_image.png & 0.14 & 0.246 & 0.801 & 0.032 & 0.854 & 0.062 & 128.353 & 646.703 \\
050\_image.png & 0.038 & 0.073 & 0.61 & 0.015 & 0.973 & 0.03 & 238.893 & 790.864 \\
051\_image.png & 0.15 & 0.262 & 0.52 & 0.085 & 0.562 & 0.148 & 31.957 & 512.533 \\
052\_image.png & 0.315 & 0.479 & 0.823 & 0.081 & 0.712 & 0.146 & 59.305 & 568.401 \\
053\_image.png & 0.14 & 0.245 & 0.646 & 0.057 & 0.964 & 0.108 & 103.668 & 558.358 \\
054\_image.png & 0.238 & 0.384 & 0.722 & 0.087 & 0.806 & 0.157 & 107.342 & 545.388 \\
055\_image.png & 0.156 & 0.27 & 0.656 & 0.064 & 0.662 & 0.116 & 84.925 & 537.276 \\
056\_image.png & 0.24 & 0.387 & 0.76 & 0.076 & 0.65 & 0.136 & 91.175 & 519.773 \\
057\_image.png & 0.184 & 0.31 & 0.749 & 0.057 & 0.533 & 0.102 & 60.21 & 326.267 \\
058\_image.png & 0.293 & 0.453 & 0.725 & 0.114 & 0.774 & 0.199 & 40.258 & 379.606 \\
059\_image.png & 0.205 & 0.341 & 0.729 & 0.07 & 0.609 & 0.125 & 61.843 & 513.448 \\
060\_image.png & 0.268 & 0.423 & 0.758 & 0.088 & 0.801 & 0.159 & 40.327 & 354.193 \\
 \hline

& 
\makecell{\textbf{Avg.}}
& 
\makecell{\textbf{Avg.}}
& 
\makecell{\textbf{Avg.}} 
& 
\makecell{\textbf{Avg.}}
& 
\makecell{\textbf{Avg.}}
& 
\makecell{\textbf{Avg.}}
& 
\makecell{\textbf{Avg.}}
& 
\makecell{\textbf{Avg.}} \\
\hline
 & ..... & ..... & ..... & ..... & ..... & .....  & .....  & .....  \\
\hline
\end{tabular}
\caption{Evaluation of semantic segmentation performed by SAM in fully automatic mode.}
\label{tab:segmentation_evaluation_SAM_automatic}
\end{table}

\section{Discussion and Conclusion}

{\color{red} 
Lorem ipsum dolor sit amet, consectetur adipiscing elit, sed do eiusmod tempor incididunt ut labore et dolore magna aliqua. Ut enim ad minim veniam, quis nostrud exercitation ullamco laboris nisi ut aliquip ex ea commodo consequat. Duis aute irure dolor in reprehenderit in voluptate velit esse cillum dolore eu fugiat nulla pariatur. Excepteur sint occaecat cupidatat non proident, sunt in culpa qui officia deserunt mollit anim id est laborum.

Lorem ipsum dolor sit amet, consectetur adipiscing elit, sed do eiusmod tempor incididunt ut labore et dolore magna aliqua. Ut enim ad minim veniam, quis nostrud exercitation ullamco laboris nisi ut aliquip ex ea commodo consequat. Duis aute irure dolor in reprehenderit in voluptate velit esse cillum dolore eu fugiat nulla pariatur. Excepteur sint occaecat cupidatat non proident, sunt in culpa qui officia deserunt mollit anim id est laborum.

Lorem ipsum dolor sit amet, consectetur adipiscing elit, sed do eiusmod tempor incididunt ut labore et dolore magna aliqua. Ut enim ad minim veniam, quis nostrud exercitation ullamco laboris nisi ut aliquip ex ea commodo consequat. Duis aute irure dolor in reprehenderit in voluptate velit esse cillum dolore eu fugiat nulla pariatur. Excepteur sint occaecat cupidatat non proident, sunt in culpa qui officia deserunt mollit anim id est laborum.
}


\subsubsection{Acknowledgements} 
The authors of this paper highly appreciate all the challenge organizers and owners for providing the public dataset to the community. We also thank Meta AI for making the source code of segment anything publicly available to the community. 

% ---- Bibliography ----
%
% BibTeX users should specify bibliography style 'splncs04'.
% References will then be sorted and formatted in the correct style.
\newpage
\bibliographystyle{splncs04}
\bibliography{ref}

\end{document}
